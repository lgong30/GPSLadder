% Template for drawing GPS figure
% Author: Long Gong
\documentclass[border=2pt]{standalone}
%%%<
\usepackage{verbatim}
%%%>
\begin{comment}
:Title: Template for GPS Figure
:Author: Long Gong

A template for GPS figure.

In some ways, this TeX script works as the "model" of our application
for visualizing a GPS simulation results.

Parameters:
    width: scalar, total length of x axis
    height: scalar, total length of y axis
    xlabels: list, labels for x axis
    ylabels: list, labels for y axis
    nodes: list of dict, nodes in the ladder, its detailed structures
           look like as follows,
            [{
               "x": <x>,
               "y": <y>
            },
            ...
            ]

Programmed in TikZ by Long Gong. Templating language is Jinja2,
templaing syntax is the default setting of Jinja2.
\end{comment}

\usepackage{tikz}
\usetikzlibrary{shapes, positioning}



\begin{comment}
%% detailed results %%
{{ detailed_results }}
\end{comment}

\begin{document}
\begin{tikzpicture}


%% place x axis label

    \draw ({{ xl }}, 1pt) -- ({{ xl }}, -1pt) node[anchor=north] {{"{"}}${{ xl }}${{"}"}};


%% place y axis label

    \draw (1pt, {{ yl }}) -- (-1pt, {{ yl }}) node[anchor=east] {{"{"}}${{ yl }}${{"}"}};



% ladder

\draw ({{ node.x }}, {{ node.y }});--
    


%% axis
\draw[thick,->] (0,0) -- ({{ width }},0) ;
\draw[thick,->] (0,0) -- (0,{{ height }});

\end{tikzpicture}
\end{document}
